\documentclass[10pt]{beamer}

\usetheme{metropolis}
\usepackage{appendixnumberbeamer}

\usepackage{booktabs}
\usepackage[scale=2]{ccicons}

\usepackage{pgfplots}
\usepgfplotslibrary{dateplot}

\usepackage{xspace}
\usepackage{chronology}
\usepackage{tabls}
\usepackage{tikz}
\usepackage{bchart}

\newcommand{\themename}{\textbf{\textsc{metropolis}}\xspace}

\hypersetup{pdfpagemode=FullScreen}

\title{Linux}
\subtitle{Eine kurze Einführung}
\date{20. Apr. 2019}
\author{Alex Neher}
\titlegraphic{\hfill\includegraphics[height=1.5cm]{img/Tux.png}}

\begin{document}

\pagestyle{empty}
	
\begin{frame}

\end{frame}

\maketitle

\begin{frame}{Inhalt}
  \setbeamertemplate{section in toc}[sections numbered]
  \tableofcontents[hideallsubsections]
\end{frame}

\section{Was ist Linux?}

\begin{frame}{Was ist Linux?}
    \begin{itemize}[<+- | alert@+>]
        \item Betriebssystem/Kernel
        \item Meist verwendet für Server
        \item Android
    \end{itemize}

\end{frame}

\section{Geschichte}
\begin{frame}{Linus Torvalds}
	\begin{columns}
		 \column{0.50\textwidth}
		 	\begin{itemize}[<+- | alert@+>]
		 		\item Finnischer Informatikstudent
		 		\item Wollte Dinge mit seinem UNIX-Computer machen, die nicht unterstützt wurden
		 		\item Entschied sich, diese Dinge selbst zu programmieren
		 		\item Vorsitzender der Linux-Foundation
		 	\end{itemize}
	 	\column{0.50\textwidth}
	 		\includegraphics[scale=0.25]{img/torvalds.png}
	\end{columns}
\end{frame}

\begin{frame}{Geschichte}
	\centering
	{\tablinesep=2ex\tabcolsep=10pt
		\begin{table}
			\begin{tabular}{r l}
				\toprule
				%TODO: Check version no. UNIX and MacOS
				1970 & UNIX 1.0 \\
				1984 & MacOS 1.0 \\
				1985 & MS DOS 3.1 \\
				1991 & Erste Linux-Version \\
				1992 & Linux wird Open Source \\
				1998 & Linux wird offiziell von Tech-Firmen unterstützt \\
				2004 & $>50\%$ aller Top500 Supercomputer laufen unter Linux \\
				2008 & Android 1.0 \\
				2017 & 100\% aller Top500 Supercomputer laufen unter Linux \\
				\bottomrule
			\end{tabular}
		\end{table}
	}
\end{frame}

\section{Open Source}

\begin{frame}{Was ist Open Source?}
	\begin{itemize}[<+- | alert@+>]
		\item Der Code des Programms ist öffentlich verfügbar
		\item Jeder kann den Code lesen, verstehen und auf allfällige Fehler/Schwachstellen überprüfen
		\item Jeder kann den Code seinen Bedürfnissen anpassen und Fehler beheben
	\end{itemize}
\end{frame}

\begin{frame}{Open Source und Sicherheit/Privatsphäre}
	\begin{itemize}[<+- | alert@+>]
		\item Jeder kann den Code auf allfällige Backdoors überprüfen
		\item Falls zu viel 'nach Hause telefoniert' wird, kann man das einfach anpassen
		\item Schwachstellen werden schnell gefunden und behoben
	\end{itemize}
\end{frame}

\section{Linux heute}

\begin{frame}{Marktanteil Desktop/Laptop}
		\includegraphics[keepaspectratio, width=\textwidth]{img/market_share.png}
\end{frame}

\begin{frame}{Marktanteil Desktop/Laptop}
\includegraphics[keepaspectratio, width=\textwidth]{img/market_share.png}

	\centering
	\begin{table}
		\begin{tabular}{l|c}
			\toprule
			\textbf{Betriebssystem} & \textbf{Marktanteil} \\
			\hline
			Windows & 87.4\% \\
			MacOS & 9.7\% \\
			Linux & 2.2\% \\
			ChromeOS & 0.33\% \\
			BSD & 0.01\% \\
			\bottomrule
		\end{tabular}
	\end{table}	
\end{frame}

\begin{frame}{Marktanteil Webserver}
	\centering
	\begin{bchart}[step=10,max=100]
		\bcbar[text=Linux]{69.9}
			\smallskip
		\bcbar[text=Windows]{30.1}
	\end{bchart}
\end{frame}

\end{document}

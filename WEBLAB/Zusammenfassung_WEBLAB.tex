\documentclass[a4paper, 11pt]{article}

\setcounter{tocdepth}{3}
\setcounter{secnumdepth}{3}

\usepackage{comment} % enables the use of multi-line comments (\ifx \fi) 
\usepackage{lipsum} %This package just generates Lorem Ipsum filler text. 
\usepackage{fullpage} % changes the margin
\usepackage[utf8]{inputenc}
\usepackage{gensymb}
\usepackage{graphicx}
\usepackage{booktabs}% http://ctan.org/pkg/booktabs
\usepackage{makecell}
\usepackage{tabularx}
\usepackage[table]{xcolor}
\usepackage{array}
\usepackage{wrapfig}
\usepackage{subcaption}
\usepackage{csquotes}
\usepackage{lscape}
\usepackage{afterpage}
\usepackage{geometry}
\usepackage{listingsutf8}
\usepackage{chngcntr}
\usepackage{multicol}
\usepackage{xcolor}


\counterwithin{figure}{section}

\geometry{a4paper, margin=1in}
\renewcommand{\figurename}{Abb.}
\renewcommand{\tablename}{Tabelle}
\newcommand{\code}[1]{\texttt{#1}}

\renewcommand*{\thead}[1]{\bfseries #1}

\renewcommand{\contentsname}{Inhalt}
\renewcommand{\listfigurename}{Abbildungsverzeichnis}

\definecolor{lightgray}{rgb}{.9,.9,.9}
\definecolor{darkgray}{rgb}{.4,.4,.4}
\definecolor{purple}{rgb}{0.65, 0.12, 0.82}
\definecolor{darkgreen}{rgb}{0.05,0.56,0.06}

\lstdefinelanguage{JavaScript}{
	keywords={typeof, new, true, false, catch, function, return, null, catch, switch, var, if, in, while, do, else, case, break},
	keywordstyle=\color{blue}\bfseries,
	ndkeywords={class, export, boolean, throw, implements, import, this},
	ndkeywordstyle=\color{darkgray}\bfseries,
	identifierstyle=\color{black},
	sensitive=false,
	comment=[l]{//},
	morecomment=[s]{/*}{*/},
	commentstyle=\color{purple}\ttfamily,
	stringstyle=\color{darkgreen}\ttfamily,
	morestring=[b]',
	morestring=[b]"
}

\lstset{frame=tlrb,
	language=JavaScript,
	aboveskip=3mm,
	belowskip=3mm,
	showstringspaces=false,
	columns=flexible,
	basicstyle={\small\ttfamily},
	numbers=left,
	numberstyle=\tiny\color{gray},
	keywordstyle=\color{blue},
	commentstyle=\color{violet},
	stringstyle=\color{darkgreen},
	breaklines=true,
	breakatwhitespace=true,
	tabsize=3
}


\begin{document}
	
\title{Zusammenfassung WEBAPP FS2018}
\author{Alex Neher}
\maketitle

\tableofcontents
\newpage
\listoffigures
\newpage

\graphicspath{{./Pictures/}}

\section{JavaScript}
Javascript, auch ECMAScript genannt ist eine clientseitige web-development Sprache. Javascript kann in HTML-Code eingebunden werden, entweder inline über \code{<script></script>}-Tags, es kann mittels \code{<script src=path/to/file.js/>} geholt werden oder direkt über EventHandler \code{<input type="checkbox" name="options" onchange="order.options.giftwrap = this.checked;">}

\subsection{Basics}
\subsubsection{Variablen definieren}

Javascript hat keine Typisierung. Das heisst, Variablen können einfach mit dem Keyword \code{var} definiert werden, ohne dass ein Datentyp spezifiziert werden muss.

\begin{lstlisting}
var a = 2;  // a ist nun eine Nummer
a = 'Jetzt bin ich ein String';
a = false;
\end{lstlisting}

\subsubsection{Objekte}
JavaScript kann auch objektbasiert programmiert werden:

\begin{lstlisting}
var bachelorModule = {
	title: "Webapplication Development",
	instructor: "Thomas Koller"
};

console.log(bachelorModule.title);  //Output: Webapplication Development
console.log(bachelorModule["instructor"]); //Output: Thomas Koller

\end{lstlisting}

Objekte sind dynamisch. Das heisst, es können zur Laufzeit noch Properties hinzugefügt oder entfernt werden:

\begin{lstlisting}
//hinzufuegen von Properties
bachelorModule.credits = 3;

//entfernen von Properties
delete bachelorModule.credits;

//Ebenfalls kann gecheckt werden, ob ein Property existiert
bachelorModule.hasOwnProperty("title"); //true
bachelorModule.hasOwnProperty("credits"); //false
\end{lstlisting}


\end{document}
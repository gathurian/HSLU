\documentclass[a4paper, 11pt]{article}

\setcounter{tocdepth}{3}
\setcounter{secnumdepth}{3}

\usepackage{comment} % enables the use of multi-line comments (\ifx \fi) 
\usepackage{lipsum} %This package just generates Lorem Ipsum filler text. 
\usepackage{fullpage} % changes the margin
\usepackage[utf8]{inputenc}
\usepackage{gensymb}
\usepackage{graphicx}
\usepackage{booktabs}% http://ctan.org/pkg/booktabs
\usepackage{makecell}
\usepackage{tabularx}
\usepackage[table]{xcolor}
\usepackage{array}
\usepackage{wrapfig}
\usepackage{subcaption}
\usepackage{csquotes}
\usepackage{lscape}
\usepackage{afterpage}
\usepackage{geometry}
\usepackage{listingsutf8}
\usepackage{chngcntr}
\usepackage{multicol}
\usepackage{xcolor}
\usepackage{pifont}


\counterwithin{figure}{section}

\AtBeginDocument{\counterwithin{lstlisting}{section}}

\geometry{a4paper, margin=1in}
\renewcommand{\figurename}{Abb.}
\renewcommand{\tablename}{Tabelle}
\newcommand{\code}[1]{\texttt{#1}}

\renewcommand*{\thead}[1]{\bfseries #1}

\renewcommand{\contentsname}{Inhalt}
\renewcommand{\listfigurename}{Abbildungsverzeichnis}

\definecolor{lightgray}{rgb}{.9,.9,.9}
\definecolor{darkgray}{rgb}{.4,.4,.4}
\definecolor{purple}{rgb}{0.65, 0.12, 0.82}
\definecolor{darkgreen}{rgb}{0.05,0.56,0.06}


\lstset{frame=tlrb,
	language=Java,
	captionpos=b,
	aboveskip=3mm,
	belowskip=3mm,
	showstringspaces=false,
	columns=flexible,
	basicstyle={\small\ttfamily},
	numbers=left,
	numberstyle=\tiny\color{gray},
	keywordstyle=\color{blue},
	commentstyle=\color{violet},
	stringstyle=\color{darkgreen},
	breaklines=true,
	breakatwhitespace=true,
	tabsize=3,
	literate=%
	{Ö}{{\"O}}1
	{Ä}{{\"A}}1
	{Ü}{{\"U}}1
	{ß}{{\ss}}1
	{ü}{{\"u}}1
	{ä}{{\"a}}1
	{ö}{{\"o}}1
}


\begin{document}

\title{Zusammenfassung MOBPRO FS2018}
\author{Alex Neher}
\maketitle

\tableofcontents
\newpage

\graphicspath{{./Pictures/}}

\section{Grundlagen}
\subsection{Komponenten}
Ein Android-App besteht aus \textit{Komponenten}. Es gibt vier verschiedene Arten von Komponenten:

\begin{description}
	\item[Activity: ] Eine View. Eine Komponente, die etwas macht und ein UI hat. (Mail-App)
	\item[Service: ] Eine Activity ohne UI. Eine Komponente, die etwas im Hintergrund ausführt. (Musikplayer im Hintergrund)
	\item[Broadcast Receiver: ] Event-Listener, der auf Broadcasts und Intents hört und
	 antwortet
	 \item[Content Provider: ] Ermöglicht den Datenaustausch zwischen Applikationen (Mail App darf Bilder von der Galerie auswählen)
\end{description}

\subsection{Android Manifest}
Alle Komponenten müssen im \textbf{Android Manifest} deklariert werden. Das Manifest enthält alle Komponenten der App, welche Intents gesendet und empfangen werden können, welche Permissions die App benötigt und so weiter

\begin{lstlisting}[language=xml, captionpos=b, caption={Einfaches Android-Manifest}]
<?xml version="1.0" encoding="utf-8"?>
<manifest xmlns:android="http://schemas.android.com/apk/res/android"
package="ch.hslu.mobpro.firstapp" >

	<application
		android:allowBackup="false"
		android:icon="@drawable/ic_launcher"
		android:label="@string/app_name"
		android:supportsRtl="true"
		android:theme="@style/AppTheme" >
		<activity android:name="ch.hslu.mobpro.firstapp.MainActivity"
			android:label="FirstApp" >
			<intent-filter>
				<action android:name="android.intent.action.MAIN" />
				<category android:name="android.intent.category.LAUNCHER" />
			</intent-filter>
		</activity>
		<activity android:name="ch.hslu.mobpro.firstapp.LifecycleLogActivity" />
		<activity android:name="ch.hslu.mobpro.firstapp.QuestionActivity" />
	</application>
</manifest>
\end{lstlisting}

\subsection{Intents}
Der Wechsel zwischen Komponenten wird mittels \textbf{Intents} realisiert. Intents sind eine \textit{offene Kommunikation}. Das heisst, der Sender weiss nicht, ob der Empfänger der Kommunikation überhaupt existiert. 

Es wird unterschieden zwischen \textit{impliziten} und \textit{expliziten} Intents. Implizite Intents rufen gezielt eine Klasse auf, während implizite Intents einfach sagen, was getan werden muss (z.B. "ruf mir einen Browser auf, aber es wird nicht spezifiziert, welchen Browser genau). \\

\begin{lstlisting}[captionpos=b, caption={Beispiel eines expliziten Intents}]
Intent myIntent = new Intent(this, Receiver.class);
intent.putExtra("msg", "Hello World");
startActivity(myIntent);
\end{lstlisting}

\begin{lstlisting}[captionpos=b, caption={Beispiel eines impliziten Intents}]
Intent browserCall = new Intent();
broserCall.setAction(Intent.ACTION_VIEW);
browserCall.setData(Uri.parse("http://www.hslu.ch));
startActivity(browserCall);
\end{lstlisting}

\begin{lstlisting}[captionpos=b, caption={Empfangen und Auswerten eines Intents}]
Intent intent = getIntent();
String msg = intent.getExtras().getString("msg");
displayMessage(msg);
\end{lstlisting}

Es kann auch asynchron eine Activity aufgerufen werden, die anschliessend ein Resultat zurückliefert, welches ausgewertet wird:

\begin{lstlisting}[captionpos=b, caption={Beispiel eines asynchronen Methodenaufrufs}]
//mainActivity
Intent intent = new Intent(this, QuestionActivity.class);
intent.putExtra("question", "Und wie läufts so mit der Androidprogrammierung?");
startActivityForResult(intent, MY_REQUEST_CODE);

//QuestionActivity
Intent answerData = new Intent();
answerData.putExtra("answer", answer);
setResult(RESULT_OK, answerData);
finish();

//MainActivity
@Override
protected void onActivityResult(int requestCode, int resultCode, Intent data) {
//resultat verarbeiten
}
\end{lstlisting}

\subsection{Lebenszyklus}
Eine App kann prinzipiell drei states haben:

\begin{description}
	\item[running: ] App läuft im Fokus und Vordergrund
	\item[paused: ] App läuft im Vordergrund aber nicht mehr im Fokus (z.B. wegen Popup)
	\item[stopped: ] App läuft im Hintergrund weiter
\end{description}

Es gibt verschiedene EventListener die bei einer Statusänderung aufgerufen werden können:

\begin{multicols}{3}
	\begin{itemize}
		\item \code{onCreate()}
		\item \code{onDestroy()}
	\columnbreak
		\item \code{onPause()}
		\item \code{onResume()}
	\columnbreak
		\item \code{onStart()}
		\item \code{onStopped()}
	\end{itemize}
\end{multicols}

\section{Benutzerschnittstellen}
\subsection{Layouts}
Layouts sind XML-Dateien, die in der \code{onCreate()}-Methode einer Activity geladen werden. Es gibt grundsätzlich drei verschiedene Layout-Optionen:

\begin{description}
	\item[Linear Layout: ] Komponenten werden in Zeilen oder Spalten linear angeordnet
	\item[Constraint Layout: ] Komponenten werden "aneinadergebunden". Man kann also sagen "Komp. A ist rechts von B und unterhalb von Komp. C"
	\item[ScrollView: ] Möglichkeit für lange Layouts, die länger sind als der Screen. Erlauben nur ein Child (z.B. LinearLayout)
\end{description}
\vspace{10px}

\noindent Das Layout kann entwder als XML definiert werden (einfacher und häufiger) oder als Java-Code. Events können direkt ins XML eingebettet werden, oder sie können über die bereits bekannte Methode des Event-Listeners in java implementiert werden.

\begin{lstlisting}[language=xml, captionpos=b, caption={Button-Definition in XML}]
<Button
	android:id="@+id/main_button_startBrowser"
	android:layout_width="fill_parent"
	android:layout_height="wrap_content"
	android:text="@string/main_buttontext_startBrowser"
	android:onClick="startBrowser" //Einbettung des Event-Listeners
	android:paddingBottom="@dimen/activity_horizontal_margin"
/>
\end{lstlisting}

\begin{lstlisting}[captionpos=b, caption={Button Event-Definition in Java}]
Button button = (Button) findViewById(R.id.main_button_startBrowser)
button.setOnClickListener(new OnClickListener(){
	@Override
	public voidonClick(View v){
		startBrowser();
	}
})
\end{lstlisting}

\subsection{Ressourcen}
Ressourcen wie Strings, Layouts, Bilder, Arrays etc. werden im \code{/res}-Ordner abgelegt. In der XML-Datei kann über den @-Operator darauf zugegriffen werden (\code{@string/value1}). In Java wird über die automatisch generierte \code{R}-Klasse auf Ressourcen zugegriffen (\code{R.layout.activity\textunderscore main}). Es können mehrere Layout- oder String-Dateien erstellt werden z.B. für Portrait und Landscape Mode oder für verschiedene Sprachen.

\subsection{Interaktion mit dem User}
\subsubsection{Options Menu}
Seit Android 3 oder so gibt es rechts oben in einer App normalerweise drei Punkte, über welche das Options-Menu aufgerufen werden kann. 

Das Options Menu-Layout wird wie alle anderen Layouts über ein XML definiert, welches anschliessned im \code{/res}-Ordner abgelegt wird. 

\begin{lstlisting}[language=xml, captionpos=b, caption={Beispiel eines Menu Layouts}]
<menu xmlns:android="http://schemas.android.com/apk/res/android"
	xmln:tools="http://schemas.android.com/tools" tools:context=".MainActivity">
	<item
		android:id="@+id/main_menu_finish"
		android:title="@string/menu_finish">
	</item>
	<item
		android:id="@+id/main_menu_values"
		android:title="@string/menu_ShowValues">
	</item>
</menu>
\end{lstlisting}

Das Menu wird anschliessend mit dem \code{MenuInflater} 'aufgeblasen'

\begin{lstlisting}[captionpos=b, caption={Beispiel des Menu-Inflators}]
@Override
pulic boolean onCreateOptionsMenu (Menu menu){
	suuper.onCreateOptoinsMenu(menu);
	MenuInflater inflater = getManuInflater();
	inflater.inflate(R.menu.menu_main, menu);
	return true;
}
\end{lstlisting}

\subsubsection{Toast}
Ein Toast ist eine kleine Meldung auf dem Bildschirm, die dem User etwas mitteilt. Der User kann aber nicht mit dem Toast interagieren.

\begin{lstlisting}[caption={Toast-Beispiel}]
Toast toast = Toast.makeText(context, "Toast Bsp", Toast.LENGTH_LONG).show();
\end{lstlisting}

\subsubsection{Dialog}
Der Dialog oder Alert ist ein Popup, der dem User eine Information mitteilt und er darauf reagieren muss.

\begin{lstlisting}[caption={Alert-Beispiel}]
AlertDialog.Builder builder = new AlertDialog.Builder(this);
builder.setTitle(R.string.dialog_fire_missiles_title)
	.setMessage(R.string.dialog_fire_missiles)
	.setPositiveButton(R.string.fire, new DialogInterface.OnClickListener() {
		public void onClick(DialogInterface dialog, int id) {
			// FIRE ZE MISSILES!
		}
	})
	.setNegativeButton(R.string.cancel, new DialogInterface.OnClickListener() {
		public void onClick(DialogInterface dialog, int id) {
			// User cancelled the dialog
		}
	});
AlertDialog dialog = builder.create();
dialog.show();
\end{lstlisting}

\subsubsection{Notifications}
Kommen später

\subsection{Adapter}
Adapter nehmen, wie bereits aus APPE/VSK bekannt, Daten, konvertieren sie in ein anderes Format und übergeben sie dem Zielkomponenten.

\begin{lstlisting}[caption={Array-Adapter um String Array in Spinner zu füllen}]
String[] myArray = new String[]{"Fanta", "Cola", "Eistee"};
ArrayAdapter<String> adapter = new ArrayAdapter<String>(this. android.R.layout.of.spinner, myArray);
this.setListAdapter(adapter);
\end{lstlisting}

Alternativ kann man sich den Adapter auch sparen und das direkt im XML des Komponenten (z.B. Spinner) machen:

\begin{lstlisting}[caption={Spinner-Layout mit direkten Füllen}]
<Spinner
	android:id="@+id/main_spinner"
	android:layout_width="match_parent"
	android:layout_height="wrap_content"
	android:entries="@array/itCourses" /> //Füllt das Array in den Spinner
</Spinner>

//in array.xml
<resources>
	<string-array name="itCourses">	
		<item>MOBPRO</item>
	</string-array>
</resources>
\end{lstlisting}

\subsection{Kontextspeicherung}
Der Zustand der App geht verloren, wenn die App gestoppt oder pausiert wird, oder auch wenn  z.B die Bildschirmorientierung geändert wird. Man kann aber bestimmte Daten kurzzeitig im Memory speichern und sie anschliessend wieder abrufen:

\begin{lstlisting}[caption={Abspeichern und Abrufen von Values im Memory}]
//Speichern
@Override
protected void onSaveInstanceState(Bundle outState){
	outState.putInt(KEY, value);
	super.onSaveInstanceState(outState);
}

//Abrufen
@Override protected void onRestoreInstanceState(Bundle savedInstanceState){
	super.onRestoreInstanceState(savedInstanceState);
	value = savedInstanceState.getInt(KEY);
}
\end{lstlisting}

\section{Persistenz}
Es wird grundsätzlich zwischen drei Arten von Präferenzen unterschieden:

\begin{description}
	\item[Default Shared Preferences: ] \code{getDefaultSharedPreferences(this)}. Für die gesamte App
	\item[Shared Preferences: ] \code{getSharedPreferences(name, mode)}. Beliebig viele Präferenzen pro App mit je einem eigenen Namen
	\item[Private Preferences: ] \code{getPreferences(mode)}. Für die aktuelle Aktivität
\end{description}

\begin{lstlisting}[caption={Holen des ResumeCount und um eins erhöht wieder abspeichern}]
final SharedPreferences preferences = getPreferences(MODE_PRIVATE);
final int newResumeCount = preferences.getInt(COUNTER_KEY, 0)+1;
final SharedPreferences.Editor editor = preferences.edit();
editor.putInt(COUNTER_KEY, newResumeCount);
editor.apply();
\end{lstlisting}

\subsection{Dateisysteme}
Dateien können privat oder öffentlich sin. Private Dateien sind ausschliesslich aus der Applikation heras oder über Content Providers zugreifbar und werden im Applikationsverzeich abgelegt. Öffentliche Daten werden auf der SD-Karte oder dem internen Filesystem abgelegt, wo jeder Zugriff drauf haben kann.

\begin{lstlisting}[caption={File schreiben}]
Writer writer = null;
try{
	writer = new BufferedWriter(new FileWriter(outFile));
	writer.write(text);
	return true;
} catch (final IOException ex){
	//catch Exception
}
\end{lstlisting}

Um aufs PUBLIC Filesystem (SD-Karte) zugreifen zu können, muss zuerst eine Berechtigung dafür erlangt werden. Alle Berechtigungen, die eine App benötigt, werden im Android Manifest festgelegt

\begin{lstlisting}[language=xml, caption={Berechtigung im Android Manifest für das Lesen vond der SD-Karte}]
<ises-permission android:name="android.permission.WRITE_EXTERAL_STORAGE"/>
\end{lstlisting}

\begin{lstlisting}[caption={Checken, ob man die Berechtigung hat und wenn nicht, Berechitung anfragen}]
int grant = checkSelfPermission(Manifest.permission.WRITE_EXTERNAL_STORAGE);
if(grant != PackageManager.PERMISSION_GRANTED) {
	requestPermissions(new String[]{Manifest.permission.WRITE_EXTERNAL_STORAGE},23);
} else {
	writeSDCard();
}
\end{lstlisting}

Die Berechtigungen werden über eine Callback-Methode ausgewertet:

\begin{lstlisting}[caption={Verarbeitung der Permission-Grants}]
public void onRequestPermissionsResult(int requestCode, String[] permissions, int[] grantResults){
	switch (requestCode){
		case 24:
			if(grantResults.length > 0 && grantResults[0] != PackageManager.PERMISSION_GRANTED){
				Toast.makeText(this, "Permission " + permissions[0] + " denied!", Toast.LENGTH_SHORT).show();
			} else {
				readSDCard();
			}
\end{lstlisting}

\subsection{SQLite}
SQLite ist ein open source relationales Datenbank-Management System, welches für Android optimiert ist. Man hat pro App beliebig viele Datenbanken, jedoch nur eine Datei pro Datenbank. Über der DB gibt es noch eine Abstraktionsebene, den sog. \textit{Room}. Der Programmierer greift jedoch nur via \code{dbAdapter()} auf die Datenbank bzw. den Room zu:

\begin{lstlisting}[caption={Anwendung des dbAdapters}]
dbAdapter = new DbAdapter(this);
dbAdapter.open();
Note note = dbAdapter.getNote(17);
\end{lstlisting}  

\section{Content Providers}
Wie bereits im ersten Kapitel erwähnt, ermöglicht der Content Provider den Austausch von Daten über Applikationsgrenzen hinweg. Dies indem er mittels eindeuten URIs auf Items (\code{content://anwendung/gruppe/item}) oder Verzeichnisse (\code{content://anwendung/gruppe})) zugreift. Das Android-OS liefert bereits einige in-house Content Provider wie z.B. Kontakte, Kalender o.ä. 

Der Zugriff auf solche Content Providers läuft immer über den \code{ContentResolver} \\ (\code{Context.getContentResolver()}) und gibt stets einen \textit{Cursor} zurück:

\begin{lstlisting}[caption={Auslesen aller SMS mittels Content Provider}]
public void showSMSList(final View view){
	final Cursor cursor = getContentResolver().query(
		Telephony.Sms.Inbox.CONTENT_URI,
		new String[]{
			Telephony.Sms.Inbox._ID,	//SMS-Id
			Telephony.Sms.Inbox.BODY	//SMS-Text
		},
		null,	//selection
		null,	//selection args
		null	//sort order
	);
}
\end{lstlisting}
\vspace{10px}

\noindent Man kann sich auch einen eigenen Content Provider schreiben, wenn man auf andere Daten zugreifen will, die nicht von den systemeigenen Providern abgedeckt werden. 

\begin{itemize}
	\item Klasse muss von \code{android.content.ContentProvider} abgeleitet sein
	\item Klasse muss bei App-Start initiiert werden (z.b. in der \code{onCreate()-Methode})
	\item CRUD-Methoden (zumindest die, die benötigt werden) müssen implementiert werden
	\item Es muss entschieden werden, ob der Provider exportiert werden soll oder nicht (exportiert = auch andere Anwendungen können auf ihn zugreifen und Daten holen)
\end{itemize}


\section{Kommunikation}


\end{document}
